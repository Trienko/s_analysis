\documentclass[a4paper,10pt]{article}
\usepackage[utf8]{inputenc}
\usepackage{amsthm,amsmath,amsfonts,amssymb,graphicx,subfigure}
\usepackage{algorithm}
\usepackage[noend]{algpseudocode}
\title{Extracting Linear Tracks from Heatmaps}
\author{T.L. Grobler}

\begin{document}
\maketitle
\begin{abstract}
In this paper 
\end{abstract}

\begin{section}{Introduction}
Extracting routes from heat-maps that were constructed from the movements of say animals, humans, vehicles or vessels is an important problem to address. Most often, when we
first inspect a dataset containing the temporal-spatial coordinates of multiple objects we would create a heatmap (grid data onto a two-dimensional grid) so that we can get a feel for the data. Once we have some understanding 
we can process and extract useful information from the original ungridded data. Normally, no automatic information is extracted from the heatmap, it merely serves as an inspection tool
which enables a datascientist to better analyze the data. In this paper, we develop a technique which can be used to automatically extract useful information from a heatmap. This extracted 
information can then be used in conjunction with the intuition gained from inspecting the heatmap to increase a datascientist's ability to analyze the data, i.e. we propose a 
heatmap segmenation algorithm. The identified segments can then be used 
to identify frequently used routes as well as  regions containing above average traffic. The identified regions can then be inspected more closely by a datascientist.
In this paper we focus on Automatic Identification System (AIS) data, which can be used to help track the movement of ships at sea.

The pseudo-code of the proposed segmentation algorithm is given in Algorithm 1. 
\begin{algorithm}
 \caption{AIS Polygon Heat-map Segmentation Algorithm}\label{euclid}
 \begin{algorithmic}[1]
 \Procedure{polygonSegmentation}{heatmap,coastlinemask,landmask}
 \State $\textrm{heatmap} \gets \log(\textrm{heatmap}+1)$
 \Comment{emphasizes the linear tracks}
 \State $\textrm{copy} \gets \textrm{heatmap}$
 \Comment{store a copy of original}
 \State $\textrm{heatmap} \gets \textrm{maskCoastline(heatmap,coastlinemask)}$
 \Comment{mask coastline}
 \State $\textrm{heatmap} \gets \textrm{heatmap - medianFilter(heatmap)}$ 
 \Comment{emphasizes the linear tracks}
 \State $\textrm{heatmap} \gets \textrm{otsuThreshold(heatmap)}$
 \Comment{binary segmentation}
 \State $\textrm{heatmap} \gets \textrm{binaryOpening(heatmap)}$
 \Comment{clears the noisy region}
 \State $\textrm{lines} \gets \textrm{houghTransform(heeatmap)}$
 \Comment{extract lines from image}
 \State $\textrm{polygons} \gets \textrm{polygonize(heatmap,coastlinemask,landmask)}$
 \For{polygon in polygons}
     \State $\textrm{heatmap[polygon]} \gets \textrm{average(copy[polygon])}$ 
 \EndFor
 \State $\textrm{heatmap[coastlinemask]} \gets \textrm{average(copy[polygon])}$
 \State $\textrm{heatmap[landmask]} \gets 0$
 \Comment{segmentation}
 \State plot(heatmap)
 
% \State $\textit{stringlen} \gets \text{length of }\textit{string}$
% \State $i \gets \textit{patlen}$
% %\BState \emph{top}:
% \If {$i > \textit{stringlen}$} \Return false
% \EndIf
% \State $j \gets \textit{patlen}$
% %\BState \emph{loop}:
% \If {$\textit{string}(i) = \textit{path}(j)$}
% \State $j \gets j-1$.
% \State $i \gets i-1$.
% \State \textbf{goto} \emph{loop}.
% \State \textbf{close};
% \EndIf
% \State $i \gets i+\max(\textit{delta}_1(\textit{string}(i)),\textit{delta}_2(j))$.
% \State \textbf{goto} \emph{top}.
 \EndProcedure
 \end{algorithmic}
 \end{algorithm}



the first thing we do is to create a heatmap of our collected data, this is to provide us with some visualization 

This document outlines a possible approach to proving that the gain solutions $\boldsymbol{g}$ can not contain a zero element. This simple idea is turning
out to be a bit harder to prove than I wish it was (maybe there is a simpler way, that I am missing). The paper is unfortunately not rigorous enough without such a proof. Altough I do believe we should submit without
this proof and try to work it into the next paper. The entire derivation works, except if $g_p = 0$ (which is exceptable ``colateral damage'' according to me). The idea is
to use the Perron-Frobenius theorem extended to complex matrices \cite{Noutsos2012}. This is a very new theorem and was derived in 2012. 

\end{section}

\end{document}
